Most materials have two regimes of operation when it comes to the relationship between stress and strain of the material, namely linear and non-linear. while phenomenon and material characterization in the linear regime of operation is pretty well understood, the non-linear regime is not as well understood. This is an intriguing part of the problem as materials that undergo plastic deformation and fatigue loading operate under this non-linear regime, and characterization of these properties help in various manufacturing processes.

This study aims to statistically model and extract relevant parameters to measure non-linearity and its effects on a material by the use of ultrasonic waves, which provide a high strain rate, but very low strain, which is ideal to test the material without changing any of its properties at the current state. We first characterize parameters through harmonics generation and then proceed to non-linear wave mixing, a technique which gives us spatial specificity in our measurements. 

The forward model was first built by creating a Finite Difference Time Domain (FDTD) solution to a set of differential equations that represent two dimensional non-linear wave propagation in an isotropic solid medium in a euclidean coordinate system. Wave mixing was simulated using a transverse and a longitudinal wave mixing in collinear path, with a phased array simulated as the transducer. Sensitivity analysis was performed for this solution and this formed the basis of our inverse model that helped predict material parameters.

The inverse model for the forward model was first built using linear regression and the results were compared with a statistical learning technique. We used Gaussian Process modelling to model the predictive model, which we further used to build the inverse model. To evaluate the model, noise was added to the measurements at various Signal to Noise Ratio (SNR) and the error percentage was measured. This model proved to be sufficient for the inverse model. From this, we could effectively estimate model parameters from wave mixing measurements.

The thesis further explores the directions this work could take along with the applications of the said techniques. By expanding on the dimensionality and complexity of the problem, we can effectively use this technique to monitor as well as improve manufacturing processes. 