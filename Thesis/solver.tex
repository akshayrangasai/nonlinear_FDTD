\begin{lstlisting}
import numpy as np
import scipy as sp
import defaults as df
from math import sin, pi, cos
from matplotlib.pyplot import imshow, plot, show, draw, pause, clim, figure
import sys
from constants import *
#from matplotlib import figure
class Solver:

    Simulation = None
    Location = None
    Width = None
    #Create a Movie Variable to calculate number of movies and plots, to bring them up when necessary. add arguments to put it in grid, instead of what's happening here. This is hardcoded waste.
    def putMovie(self, pauseTime):
        data = np.reshape(self.Simulation.Grid[:,:,0,1],  (self.Simulation.ElementSpan[0],self.Simulation.ElementSpan[1]))       #       # 
        figure("Wave Movie Transverse")
        imshow(data)
        clim([-1e-8,1e-8])
        draw()
        pause(pauseTime)
        
        data = np.reshape(self.Simulation.Grid[:,:,1,1],  (self.Simulation.ElementSpan[0],self.Simulation.ElementSpan[1]))       #       # 
        figure("Wave Movie Longitudinal")
        imshow(data)
        clim([-1e-8,1e-8])
        draw()
        pause(pauseTime)
 
       
            
    def putSource(self, i, frequency, index, waveType = 0):
        #Multiply with gaussian to remove edge effects.
        
        #Adding default Source
        
        #waveType 0 - Transverse. 1 - Longitudinal
        
        X_S = round(self.Location[0]) 
        Y_S = slice(round(self.Simulation.ElementSpan[0]/2) - round(self.Width[1]/2), round(self.Simulation.ElementSpan[0]/2) + round(self.Width[1]/2))
      
        #Raised Cosine Pulse.
        
        self.Simulation.Grid[Y_S,index,waveType,2] = (1-cos(2*pi*frequency*i*self.Simulation.Dt/self.Simulation.Pulses))*cos(2*pi*frequency*i*self.Simulation.Dt)*1e-8
        
        #Sine Pulse. Trying to recreate the paper
        #self.Simulation.Grid[Y_S,index,waveType,2] = sin(2*pi*frequency*i*self.Simulation.Dt)*1e-8

        #print self.Simulation.Grid[X_S, round(self.Simulation.ElementSpan[1]/2) - round(self.Width[0]/2) + 1,0,2]
            
    #Line Sources only, currently. Multiple Sources must be accounted for, Must think of a matrix solution. So much fight for something that might not even work. Pain.
    def setSource(self, Location = None, Width = None, Theta = None) :
        
        if Location is None:
            self.Location = [df.LOCATION*self.Simulation.Dimensions[0]/self.Simulation.Dx]
        else:
            self.Location.append(Location*self.Simulation.Dimensions[0]/self.Simulation.Dx)

        if Width is None:
            self.Width = [(df.WIDTH/self.Simulation.Dx)*D for D in self.Simulation.Dimensions]
        else:
            self.Width.append((Width/self.Simulation.Dx)*D for D in self.Simulation.Dimensions)
            
        if Theta is None:
            self.Location = [df.THETA]
        else:
            self.Location.append(Theta)

        
    
    def Solve(self):
        #First Equation We'll be solving will be the standard wave equation.
        self.setSource()
        #Setting the Source first. Now, let's solve the DE like a boss
        
        _X = slice(0,self.Simulation.ElementSpan[0]-2)
        X = slice(1,self.Simulation.ElementSpan[0]-1)
        X_ = slice(2,self.Simulation.ElementSpan[0])
        
        _Y = slice(0,self.Simulation.ElementSpan[1]-2)
        Y = slice(1,self.Simulation.ElementSpan[1]-1)
        Y_ = slice(2,self.Simulation.ElementSpan[1])

        #_X indicates previous X coordinate and X_ indicts the one after
        r_var = round(self.Simulation.Time/self.Simulation.Dt)
        #print "Total Iterations are " , r_var
        c_t2 = pow(self.Simulation.MaterialProperties.WaveVelocityT,2)
        c_l2 = pow(self.Simulation.MaterialProperties.WaveVelocityL,2)

       # sdata = sp.zeros((r_var,1))
        for i in range(1,int(r_var)):
            dv_y = (self.Simulation.Grid[X,Y_,1,1] - self.Simulation.Grid[X,_Y,1,1])/(2*self.Simulation.Dx)
            d2v_y = (self.Simulation.Grid[X,Y_,1,1] - 2*self.Simulation.Grid[X,Y,1,1] + self.Simulation.Grid[X,_Y,1,1])/pow(self.Simulation.Dx,2)
            du_y = (self.Simulation.Grid[X,Y_,0,1] - self.Simulation.Grid[X,_Y,0,1])/(2*self.Simulation.Dx)
            d2u_y = (self.Simulation.Grid[X,Y_,0,1] - 2*self.Simulation.Grid[X,Y,0,1] + self.Simulation.Grid[X,_Y,0,1])/pow(self.Simulation.Dx,2)

            #Solving for Displacements in the X directio
            
            self.Simulation.Grid[X,Y,0,2] = 2*self.Simulation.Grid[X,Y,0,1] - self.Simulation.Grid[X,Y,0,0] + pow(self.Simulation.Dt,2)*(c_t2*d2u_y + self.Simulation.MaterialProperties.BetaT*c_t2*(dv_y*d2u_y + du_y*d2v_y))
            self.Simulation.Grid[X,Y,1,2] = 2*self.Simulation.Grid[X,Y,1,1] - self.Simulation.Grid[X,Y,1,0] + pow(self.Simulation.Dt,2)*(c_l2*d2v_y + self.Simulation.MaterialProperties.BetaL*c_l2*dv_y*d2v_y + self.Simulation.MaterialProperties.BetaT*c_t2*du_y*d2u_y)
 
            
            
            self.Simulation.SourceSignal[i,0] = sum(self.Simulation.Grid[:,-2,0,2])/self.Simulation.Grid.shape[1]
             
            
            self.Simulation.SData[i,0] = sum(self.Simulation.Grid[:,1,1,2])/self.Simulation.Grid.shape[1] 
            
            #self.Simulation.SData[i,0] =  sum(self.Simulation.Grid[:,1,0,2])/self.Simulation.Grid.shape[1]  
            
            #print self.Simulation.Grid[15,15,0,2]
            
            #Boundary COnditions. Making the ends soft reflections. Let's see how that works out.

            '''
            if self.Simulation.Mixing is not True:
                self.Simulation.Grid[-1,:,0,2] = self.Simulation.Grid[-2,:,0,2]
            else:
            #    self.Simulation.Grid[:,0,1,2] = self.Simulation.Grid[:,1,1,2]
            #    self.Simulation.Grid[:,0,0,2] = self.Simulation.Grid[:,1,0,2]
                self.Simulation.Grid[:,-1,0,2] = self.Simulation.Grid[:,-2,0,2]
            #    self.Simulation.Grid[:,-2,1,2] = self.Simulation.Grid[:,-1,1,2]

             #Updates go Here
            '''

                                           
            if(i <= round(self.Simulation.Pulses*(1.0/(self.Simulation.WaveProperties.Frequency))/self.Simulation.Dt)):
                self.putSource(i,self.Simulation.WaveProperties.Frequency, 0, TRANSVERSE)
            else:
                self.Simulation.Grid[:,1,0,2] = self.Simulation.Grid[:,0,0,2]
                self.Simulation.Grid[:,-2,0,2] = self.Simulation.Grid[:,-1,0,2]
                #self.Simulation.Grid[:,0,1,2] = self.Simulation.Grid[:,1,1,2]
            
            
            if self.Simulation.Mixing == True:
                
                if(i <= round(self.Simulation.Pulses*(1.0/(0.997*4*self.Simulation.WaveProperties.Frequency))/self.Simulation.Dt)):
                    self.putSource(i,0.997*4*self.Simulation.WaveProperties.Frequency,-1,LONGITUDINAL)
                else: 
                    self.Simulation.Grid[:,-1,1,2] = self.Simulation.Grid[:,-2,1,2]
                    self.Simulation.Grid[:,0,1,2] = self.Simulation.Grid[:,1,1,2]
                    #self.Simulation.Grid[:,-1,0,2] = self.Simulation.Grid[:,-2,0,2]
 
            self.Simulation.Grid[:,:,1,0] = self.Simulation.Grid[:,:,1,1]
            self.Simulation.Grid[:,:,1,1] =  self.Simulation.Grid[:,:,1,2]
            
            self.Simulation.Grid[:,:,0,0] = self.Simulation.Grid[:,:,0,1]
            self.Simulation.Grid[:,:,0,1] =  self.Simulation.Grid[:,:,0,2]
#            print i
            if i%round(0.05*r_var) == 0:
                #print i
                if self.Simulation.ViewMovie == True:
                    self.putMovie(0.01)    
                sys.stdout.write('=='*int(round(i/round(0.1*r_var))))
                
                #p.plot.show()
        #print self.Simulation.MaterialProperties.BetaL, self.Simulation.MaterialProperties.BetaT, self.Simulation.MaterialProperties.WaveVelocityL, self.Simulation.Dt
        
    '''
        figure("Source Signal")
        plot(self.Simulation.SourceSignal)

        pause(0.01) 
        figure("Non Linear Signal")
        plot(self.Simulation.SData)
        show()
    '''
        
#        np.save("TotalSignal",self.Simulation.SourceSignal) 
#        np.save("LinSignal",sdata)
    def __init__(self, Simulation = None):
        if Simulation is None:
            raise ValueError("Simulation Cannot be None. Please Initialize a New Simulation to proceed")
        else:
            self.Simulation = Simulation  
            self.Solve()

if __name__ == "__main__":
    raise Exception("Cannot run file as a standalone file. Please run through proper initialized channels")
\end{lstlisting}