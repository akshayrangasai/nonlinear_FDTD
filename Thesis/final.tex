\section{Conclusion}
The estimation of higher order coefficients from amplitude data of collinear wave mixing is  a problem for which an inversion exists and thus, is solvable. During the course of the work undertaken, the forward problem proved to be more challenging than the inverse problem, to solve. This is simply because of the techniques employed for inversion. The inverse model and its results have shown less than an 8\% error in prediction, and thus is accurate within limits of acceptability.

The techniques used in the solution of this problem are taken from computer science and applied to this specific problem. The results have been quite encouraging, and this must be looked into further. The work done can act as the basis of much more advanced studies, and this will be explored further below. 

\section{Future Work}
The work described in this project can be extended further by using this as a strong foundation. This forms the framework for future work, which is exciting and has many real world applications. The specific case we have solved here, is one of the many cases of a generic problem. Added to that fact that it is constrained to the 2-Dimensional space currently, there is huge potential to build on this work.

\subsection{Theoretical}
Upgrading this from the 2D space to the 3D space is something that can be worked on. While this problem is non-trivial, it could prove to be challenging due to the sheer computational volume involved. With a 3D implementation, the possibilities of this simulation are endless.\cite{noncollinear} \cite{phd_thesis}

Multiple variations of this problem can be worked on, from going from an L-T wave mixing to a generic wave mixing scheme. The solver can be updated to solve for any sort of wave excitation. This will result in a much greater variety of solutions and possibilities in the future. With different wave mixing, other parameters can also be extracted from a material.

Finally, a non-collinear scheme can be developed for a similar use case. A non-collinear scheme is not only more directional but also localized in its solutions. This helps us analyze material and estimate parameters with extremely high accuracy. The trade off for such convenience is highly correlated variables which makes the inversion more involved.

Working in a similar vein, a manufacturing process can be simulated, for example, forming and the effects of the process on these constants can be estimated using a wave mixing technique. This could help optimise process parameters to get much better results and greater efficiency.

\subsection{Experimental}

Model validation can be pursued experimentally with a sample of a highly strained piece of material. While there is literature which has explored this path, the inversion from said data isn't something that has been carried out by them.

The technique can further be applied to specimens that undergo intense stress in manufacturing processes to understand said processes better. At multiple sages of forming, specimen can be evaluated to determine the constants values and this can be used to model the process better.

Non-collinear techniques can be experimented with. These techniques are more specific and local and grant the experimenter the freedom to set up his apparatus comfortably. For non-regular shapes, non-collinear wave mixing approach works better than a collinear one as it is more directional.

The references given on this page will direct the user to the potential and areas of work that can be focused on in the future.